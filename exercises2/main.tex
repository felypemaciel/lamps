\documentclass[11pt,a4paper]{article}

\usepackage[english]{babel}
\usepackage[T1]{fontenc}
%\usepackage[latin1]{inputenc}

\usepackage{amsmath}
\usepackage{multirow}
%\usepackage[margin=2cm]{geometry}
\usepackage[dvipsnames]{xcolor}

\usepackage{graphicx}
\usepackage{amssymb}
\usepackage{mathrsfs}
\usepackage{subcaption}
\usepackage{setspace} 
\usepackage{enumitem}

\definecolor{AB}{rgb}{0,0.3961,0.7412}

\usepackage[bookmarksnumbered=true]{hyperref} 
\hypersetup{
     colorlinks = true,
     linkcolor = AB,
     anchorcolor = AB,
     citecolor = AB,
     filecolor = AB,
     urlcolor = AB
 }
\usepackage{enumitem}
\allowdisplaybreaks

\addtolength{\oddsidemargin}{-.8in}
\addtolength{\evensidemargin}{-.65in}
\addtolength{\textwidth}{1.68in}
\addtolength{\topmargin}{-.875in}
\addtolength{\textheight}{1.15in}

\usepackage{titlesec}
\titleformat{\section}{\fontsize{12}{14}\sc}{\thesection}{1em}{}[]
\titleformat{\subsection}{\fontsize{11}{12}\sc}{\thesubsection}{1em}{}[]
\titleformat{\subsubsection}{\fontsize{10}{11}\sc}{\thesubsection}{1em}{}[]
 
 \setlength{\parindent}{0pt}
 
\usepackage{tocloft}
\renewcommand{\cfttoctitlefont}{\sc}
\renewcommand{\cftsecfont}{\sc}
\renewcommand{\cftsubsecfont}{\sc}

\usepackage{abstract}
\renewcommand\abstractnamefont{\sc}

\usepackage[round]{natbib}

 
\begin{document}
\thispagestyle{empty}

\begin{flushleft} \includegraphics[width=1.0\textwidth]{headers}   \end{flushleft}

\vskip0.5cm
\begin{spacing}{2} {\Large\sc\noindent Exercise List 2}
\end{spacing}

\vfill

\begin{spacing}{1.2}
{\large\sc  Antonio Felype Ferreira Maciel - 576261}\\

{\noindent \large \sc Master's Course in Teleinformatics Engineering}\\
{\noindent \large \sc Federal University of Ceará}\\

{\noindent \large \sc TIP8300 - Nonlinear System Optimization}
\end{spacing}

\newpage 

% \tableofcontents

\newpage

%\begin{abstract}
% This report provides a concise introduction to Up-flow Anaerobic Sludge Blanket reactor, outlines its applications in Brazil, and explains the model we are using for simulations. We discuss our intentions for using this model, the challenges we are currently facing and our next steps. 
%\end{abstract}

% \section{Introduction} \label{sec:intro}

\begin{enumerate}
    \item Considere o seguinte problema de otimização:
        \begin{align*}
            \text{minimize} & \quad (x_1 - 1)^2 + (x_2 - 2)^2\\
            \text{sujeito a} & \quad (x_1 - 1)^2 = 6x_2 
        \end{align*}
        \begin{itemize}
            \item[(a)] Como este problema pode ser classificado?
            
            O problema é um Programa Quadrático com Restrição Quadrática (QCQP). 

            \item[(b)] Verifique se é possível encontrar um problema equivalente convexo.
            
            Utilizando a relaxação convexa, podemos relaxar a restrição de igualdade para uma restrição de inequidade:
            \begin{equation*}
                (x_1 - 1)^2 \leq 6x_2.
            \end{equation*}

            Dessa forma, o problema passa a ser um problema de otimização convexo.

            \item[(c)] Expresse as condições de KKT para o problema.
            
            Neste problema, desejamos minimizar a função $f(x_1, x_2) = (x-1)^2 + (x_2 - 2)^2$ sujeita à restrição $h(x_1, x_2) = (x_1 - 1)^2 - 6x_2 = 0$.

            As condições de KKT para o problema são:

            \begin{enumerate}[label=\Roman*]
                \item \textbf{Restrições primárias:} $f_i(x) \leq 0, \, i=1,\dots, m, \, h_i(x) = 0, \, i=1, \dots, p$.
                \begin{equation}
                    h(x_1,x_2) = 0 \leadsto (x_1 - 1)^2 - 6x_2 = 0
                    \label{eq:primal-constraint}
                \end{equation}
                \item \textbf{Restrições duais:} $\lambda \succeq 0$.
                
                Para as restrições de igualdade, não há restrições de sinal para o multiplicador de Lagrange. Logo, a restrição dual é $\lambda \in \mathbf{R}$.
                \item \textbf{Folga complementar:} $\lambda_i f_i(x) = 0, \, i=1, \dots, m$.
                
                Para restrições de igualdade, a folga complementar é satisfeita já que a restrição é rígida.

                \item O gradiente do Lagrangeano com respeito a $x$ desaparece:
                \begin{equation*}
                    \nabla f_0(x) + \sum_{i=1}^{m} \lambda_i \nabla f_i(x) + \sum_{i=1}^{p} \nu_i h_i(x) = 0.
                \end{equation*}

                A função Lagrangeano $\mathcal{L}(x_1, x_2, \lambda)$ pode ser definida como:
                \begin{equation*}
                    \mathcal{L}(x_1, x_2, \lambda) = f(x_1, x_2) + \lambda \cdot h(x_1,x_2).
                \end{equation*}
                
                Substituindo $f(\cdot)$ e $h(\cdot)$ em $\mathcal{L}(\cdot)$, temos:
                \begin{equation*}
                    \mathcal{L}(x_1, x_2, \lambda) = (x_1 - 1)^2 + (x_2 - 2)^2 + \lambda((x_1-1)^2 - 6x_2).
                \end{equation*}

                O gradiente do Lagrangeano com respeito a $x_1$ e $x_2$ deve ser zero, então $\cfrac{\partial \mathcal{L}}{\partial x_1} = 0$ e $\cfrac{\partial \mathcal{L}}{\partial x_2} = 0$.
                \begin{equation*}
                    \begin{aligned}
                        \cfrac{\partial \mathcal{L}}{\partial x_1} = & 2(x_1 - 1)(1 + \lambda)\\
                        \cfrac{\partial \mathcal{L}}{\partial x_2} = & 2(x_2-2) - 6\lambda
                    \end{aligned}
                \end{equation*}

                As condições do gradiente são, portanto:
                    \begin{align}
                        2(x_1 - 1)(1 + \lambda) = 0 \label{eq:grad-cond1}\\
                        2(x_2 - 2) - 6\lambda = 0 \label{eq:grand-con2}.
                    \end{align}
            \end{enumerate}
            \item[(d)] Determine a solução ótima.
            
            A primeira condição do gradiente (\ref{eq:grad-cond1}) nos dá duas possibilidades: ou $x_1 = 1$ ou $\lambda = -1$. Pela segunda condição do gradiente (\ref{eq:grand-con2}), $x_2 = 2 + 3\lambda$. Substituindo na restrição primária (\ref{eq:primal-constraint}):
            \begin{itemize}
                \item Para $x_1 = 1$, $x_2 = 0$. Já que $x_2 = 2 + 3\lambda$, então $\lambda = -\cfrac{2}{3}$. Logo, uma das soluções é:
                \begin{equation*}
                    x_1 = 1, \quad x_2 = 0, \quad \lambda = -\cfrac{2}{3}.
                \end{equation*}
                \item Para $\lambda = -1$. Já que $x_2 = 2 + 3\lambda$, então $x_2 = -1$. Substituindo na restrição primária (\ref{eq:primal-constraint}), temos que
                \begin{equation*}
                    (x_1 - 1)^2 = -6,
                \end{equation*}
                \noindent o que não é possível, já que o lado esquerdo da igualdade é não-negativo, enquanto que o lado direito é negativo.
            \end{itemize}

            Portanto, a única solução que satisfaz as condições de KKT é:
            \begin{equation*}
                x_1 = 1, \quad x_2 = 0, \quad \lambda = -\cfrac{2}{3}.
            \end{equation*}
        \end{itemize}
    \item Considere o problema de otimização:
        \begin{align*}
            \text{minimize} & \quad x^2 + 3x + 1\\
            \text{sujeito a} & \quad (x - 1)(x - 3) \leq 0
        \end{align*}
        \begin{itemize}
            \item[(a)] Expresse as condições de KKT para o problema.
            
            Seja $f(x) = x^2 + 3x + 1$ a função a ser minimizada e $h(x) = (x-1)(x-3) \leq 0$ a restrição do problema. Reescrevendo $h(x)$, temos que
            \begin{equation*}
                h(x) = x^2 - 4x + 3 \leq 0.
            \end{equation*}

            O a função Lagrangeano então é dada por 
            \begin{equation*}
                \begin{aligned}
                    \mathcal{L}(x, \lambda) = & f(x) + \lambda \cdot h(x)\\
                    \mathcal{L}(x, \lambda) = & x^2 + 3x + 1 + \lambda (x^2 - 4x + 3).
                \end{aligned}
            \end{equation*}

            As condições de KKT para o problema são:
            \begin{enumerate}[label=\Roman*]
                \item \textbf{Restrição primária:} 
                \begin{equation}
                    h(x) \leq 0 \leadsto (x-1)(x-3) \leq 0
                    \label{eq:primal-constraint-2}
                \end{equation}.
                \item \textbf{Restrição dual:} 
                \begin{equation}
                    \lambda \geq 0.
                    \label{eq:dual-constraint-2}   
                \end{equation}
                \item \textbf{Folga complementar}: 
                
                \begin{equation}
                    \lambda \cdot h(x) = 0 \leadsto \lambda \cdot (x-1)(x-3) = 0
                    \label{eq:complementary-slackness-2}
                \end{equation}.
                \item \textbf{Gradiente do Lagrangeano desaparece:} 
                
                $\cfrac{\partial\mathcal{L}}{\partial x} = 0$.

                \begin{equation}
                    \begin{aligned}
                        \cfrac{\partial \mathcal{L}}{\partial x} = & \cfrac{\partial}{\partial x} x^2 + 3x + 1 + \lambda(x^2 - 4x + 3)\\
                        \cfrac{\partial \mathcal{L}}{\partial x} = & 2x + 3 + \lambda(2x - 4) = 0.
                    \end{aligned}
                    \label{eq:gradient-condition-2}
                \end{equation}
            \end{enumerate}
            \item[(b)] Determine a solução ótima.

            Utilizando a folga complementar (\ref{eq:complementary-slackness-2}), temos três possibilidades: $\lambda = 0$, $x = 1$ ou $x = 3$.
            \begin{itemize}
                \item Para $\lambda = 0$, utilizando a condição do gradiente (\ref{eq:gradient-condition-2}):
                \begin{equation*}
                    2x + 3 = 0 \leadsto x = - \cfrac{3}{2}.
                \end{equation*}
                Substituindo na restrição primária (\ref{eq:primal-constraint-2}):
                \begin{equation*}
                    (x-1)(x-3) = \cfrac{45}{4} > 0
                \end{equation*}
                \noindent o que viola a restrição primária. Portanto, $\lambda = 0$ não é uma solução válida.
                \item Para $x = 1$, utilizando a condição do gradiente (\ref{eq:gradient-condition-2}):
                \begin{equation*}
                    2 + 3 + \lambda(-2) = 0 \leadsto \lambda = \cfrac{5}{2},
                \end{equation*}
                \noindent o que está de acordo com a restrição dual (\ref{eq:dual-constraint-2}).
                \item Para $x = 3$, utilizando a condição do gradiente (\ref{eq:gradient-condition-2}):
                \begin{equation*}
                    9 + 2 \lambda = 0 \leadsto \lambda = -\cfrac{9}{2},
                \end{equation*}
                \noindent o que viola a restrição dual (\ref{eq:dual-constraint-2}). Portanto, $x = 3$ não é uma solução válida. 
            \end{itemize}
            Dessa forma, a solução ótima para o problema é:
            \begin{equation*}
                x = 1, \quad \lambda = \cfrac{5}{2}.
            \end{equation*}
            \item[(c)] Encontre a função dual e o problema dual.
            
            A função dual $g(\lambda)$ é encontrada ao minimizar o Lagrangeano com relação a x:
            \begin{equation*}
                g(\lambda) = \underset{x}{\min} \, \mathcal{L}(x, \lambda).
            \end{equation*}
            Derivando $\mathcal{L}(\cdot)$ e igualando a 0, temos
            \begin{equation*}
                \cfrac{\partial \mathcal{L}}{\partial x} = 2(1+\lambda)x + (3-4\lambda) = 0,
            \end{equation*}
            \noindent o que nos dá
            \begin{equation*}
                x = \cfrac{4\lambda - 3}{2(1+\lambda)}.
            \end{equation*}
            Substituindo $x$ no Lagrangeano, temos que a função dual é
            \begin{equation*}
                g(\lambda) = -\cfrac{(4\lambda - 3)^2}{4(1+\lambda)} + 1 + 3\lambda.
            \end{equation*}

            O problema dual é definido então como maximizar a função dual sujeito à restrição dual. Dessa forma, o problema dual é
            \begin{equation*}
                \begin{aligned}
                    \max \quad & g(\lambda)\\
                    \text{s.t.} \quad & \lambda \geq 0. 
                \end{aligned}
            \end{equation*}
            \item[(d)] Verifique se o problema apresenta dualidade forte.
            
            A função objetivo $f(x)$ é quadrática e convexa. A restrição $h(x) = (x-1)(x-3) \leq 0$ pode ser reescrita como $h(x) = x^2 - 4x + 3 \leq 0$ e também é convexa. A região viável $h(x) \leq 0$ é o intervalo $[1,3]$, que é um conjunto convexo. Dessa forma, o problema é um problema de otimização convexo. 

            De acordo com a condição de Slater, a dualidade forte é atendida se existe um ponto viável $x$ tal que $h(x) < 0$. A restrição $h(x) = (x-1)(x-3) \leq 0$ é satisfeita por $x \in [1,3]$. Para $x = 2$, $h(x) = -1 \leq 0$. Portanto, a condição de Slater é atendida.

            O problema primário pode ser resolvido encontrando o mínimo de $f(x)$ sujeito a $x \in [1,3]$. O ponto mínimo de $f(x)$ pode ser encontrado quando a derivada é zero:
            \begin{equation*}
                f'(x) = 2x + 3 = 0 \leadsto x = -\cfrac{3}{2}.
            \end{equation*}
            Porém, esse ponto não é viável pois está fora do intervalo $[1,3]$. Para $x=1$, $f(x) = 5$. Para $x = 3$, $f(x) = 19$. O valor mínimo de $f(x)$ na região viável é 5, encontrado para $x=1$.

            O problema dual é
            \begin{equation*}
                \begin{aligned}
                    & \max \, g(\lambda) = -\cfrac{(4\lambda - 3)^2}{4(1 + \lambda)} + 1 + 3\lambda\\
                    & \text{s.t.} \, \lambda \geq 0.
                \end{aligned}
            \end{equation*}

            Pelas condições de KKT, a solução ótima dual é $\lambda = \cfrac{5}{2}$. Substituindo $\lambda$ na função dual, temos que
            \begin{equation*}
                g(\cfrac{5}{2}) = 5.
            \end{equation*}

            Já que as soluções primárias e dual são iguais, a dualidade forte é válida para o problema. 

        \end{itemize}
    \item Considere o problema de otimização:
        \begin{align*}
            \text{minimize} & \quad x^2 + y^2\\
            \text{sujeito a} & \quad \begin{cases}
                x + y \geq 1\\
                y \leq 2\\
                y^2 \geq x
            \end{cases}
        \end{align*}
        \begin{itemize}
            \item[(a)] Expresse as condições de KKT para o problema.
            Seja $f(x,y) = x^2 + y^2$ a função a ser minimizada e h(x,y) as restrições do problema:
            \begin{equation*}
                \begin{cases}
                    h_1(x,y) = 1 - x - y \leq 0\\
                    h_2(x,y) = y - 2 \leq 0\\
                    h_3(x,y) = x - y^2 \leq 0.
                \end{cases}
            \end{equation*}
            
            O Lagrangeano $\mathcal{L}(x, y, \lambda_1, \lambda_2, \lambda_3)$ é definido por
            \begin{equation*}
                \mathcal{L}(x,y,\lambda) = f(x,y) + \lambda_1h_1(x,y) + \lambda_2h_2(x,y) + \lambda_3h_3(x,y).
            \end{equation*}

            Substituindo $f(x,y)$ e as restrições na equação acima, temos que
            \begin{equation*}
                \mathcal{L}(x,y,\lambda) = x^2 + y^2 + \lambda_1(1-x-y) + \lambda_2(y-2) + \lambda_3(x-y^2).
            \end{equation*}

            \begin{enumerate}[label=\Roman*]
                \item \textbf{Condições do gradiente do Lagrangeano}
                \begin{equation*}
                    \cfrac{\partial \mathcal{L}}{\partial x} = 0 \quad \text{e} \quad \cfrac{\partial \mathcal{L}}{\partial y} = 0.
                \end{equation*}
                \begin{align}
                    \cfrac{\partial \mathcal{L}}{\partial x} = & 2x - \lambda_1 + \lambda_3 = 0\label{eq:gradcond-1-q4}\\
                    \cfrac{\partial \mathcal{L}}{\partial y} = & 2y - \lambda_1 + \lambda_2 - 2\lambda_3y = 0\label{eq:gradcond-2-q4}
                \end{align}
                \item \textbf{Restrições primárias}
                \begin{equation*}
                    h_1(x,y) \leq 0, \quad h_2(x,y) \leq 0, \quad h_3(x,y) \leq 0.
                \end{equation*}
                \begin{align}
                        1 - x - y \leq 0\label{eq:restprimal1-q4}\\
                        y - 2 \leq 0\label{eq:restprimal2-q4}\\
                        x - y^2 \leq 0\label{eq:restprimal3-q4}.
                \end{align}
                \item \textbf{Restrições duais}
                \begin{equation*}
                    \lambda_1 \geq 0, \quad \lambda_2 \geq 0, \quad \lambda_3 \geq 0.
                \end{equation*}
                \item \textbf{Folga complementar}
                \begin{equation*}
                    \lambda_1h_1(x,y) = 0, \quad \lambda_2h_2(x,y) = 0, \quad \lambda_3h_3(x,y) = 0.
                \end{equation*}
                \begin{align}
                        \lambda_1(1 - x - y) = 0\label{eq:folga1-q4}\\
                        \lambda_2(y-2) = 0\label{eq:folga2-q4}\\
                        \lambda_3(x - y^2) = 0\label{eq:folga3-q4}.
                \end{align}
            \end{enumerate}

            \item[(b)] Determine os pontos que satisfazem as condições de KKT e encontre a solução ótima.
            
            Considerando as condições de folga complementar:
            \begin{enumerate}[label=\roman*]
                \item $\lambda_1 = 0$, $\lambda_2 = 0$, $\lambda_3 = 0$.
                Substituindo nas condições do gradiente (\ref{eq:gradcond-1-q4}, \ref{eq:gradcond-2-q4}), tem-se que
                \begin{equation*}
                    \begin{aligned}
                        & 2x = 0 & \leadsto \quad & x = 0\\
                        & 2y = 0 & \leadsto \quad & y = 0.
                    \end{aligned}
                \end{equation*}
                Isso porém não atende às restrições primárias e portanto não é viável: $1 - 0 - 0 = 1 \leq 0$.
                \item $\lambda_1 > 0$, $\lambda_2 = 0$, $\lambda_3 = 0$.
                
                A partir de \ref{eq:folga1-q4}:
                \begin{equation*}
                    1 - x - y = 0 \leadsto x + y = 1.
                \end{equation*}
                Substituindo em \ref{eq:gradcond-1-q4} e \ref{eq:gradcond-2-q4}:
                \begin{equation*}
                    \begin{aligned}
                        2x - \lambda_1 = 0 \leadsto \lambda_1 = 2x\\
                        2y - \lambda_1 = 0 \leadsto \lambda_1 = 2y.
                    \end{aligned}
                \end{equation*}
                Dessa forma, $2x =2y \therefore x = y$. Substituindo $x = y$ em $x + y = 1$:
                \begin{equation*}
                    2x = 1 \leadsto x = \frac{1}{2}, \quad y = \cfrac{1}{2}.
                \end{equation*}
                No entanto, o caso não é viável pois não atende à restrição primária (\ref{eq:restprimal3-q4}):
                \begin{equation*}
                    x - y^2 = \cfrac{1}{2} - \cfrac{1}{4} = \cfrac{1}{4} \leq 0.
                \end{equation*}
                \item $\lambda_1 = 0$, $\lambda_2 > 0$, $\lambda_3 = 0$
                A partir de (\ref{eq:folga2-q4}):
                \begin{equation*}
                    y - 2 = 0 \leadsto y = 2.
                \end{equation*}
                Substituindo em (\ref{eq:gradcond-1-q4}) e (\ref{eq:gradcond-2-q4}):
                \begin{equation*}
                    \begin{aligned}
                        & 2x + \lambda_3 = 0 \leadsto \lambda_3 = -2x\\
                        & 2 y - \lambda_1 + \lambda_2 - 2\lambda_3y = 0 \leadsto 4 - \lambda_1 + \lambda_2 - 4\lambda_3 = 0.
                    \end{aligned}
                \end{equation*}
                Já que $\lambda_1 = \lambda_3 = 0$
                \begin{equation*}
                    4 + \lambda_2 = 0 \leadsto \lambda_2 = -4,
                \end{equation*}
                o que viola a restrição dual. Dessa forma, o caso não é viável.
                \item $\lambda_1 = 0$, $\lambda_2 = 0$, $\lambda_3 > 0$
                A partir de (\ref{eq:folga3-q4}):
                \begin{equation*}
                    x - y^2 = 0 \leadsto x = y^2.
                \end{equation*}
                Substituindo em (\ref{eq:gradcond-1-q4}) e (\ref{eq:gradcond-2-q4}):
                \begin{equation*}
                    \begin{aligned}
                        &2x + \lambda_3 = 0 \leadsto \lambda_3 = 2x\\
                        &2y - \lambda_1 + \lambda_2 - 2\lambda_3y = 0 \leadsto y(2 + 4x) = 0.
                    \end{aligned}
                \end{equation*}
                Já que $y \neq 0$, pois $x = y^2$, tem-se que
                \begin{equation*}
                    2 + 4x = 0 \leadsto x = -\cfrac{1}{2},
                \end{equation*}
                \noindent o que viola $x = y^2 \geq 0$. Portanto, o caso não é viável.
                \item $\lambda_1 > 0$, $\lambda_2 > 0$, $\lambda_3 = 0$
                A partir de (\ref{eq:folga1-q4}) e (\ref{eq:folga2-q4}):
                \begin{equation*}
                    \begin{aligned}
                        & 1 - x - y = 0 \leadsto x + y = 1\\
                        & y - 2 = 0 \leadsto y = 2.
                    \end{aligned}
                \end{equation*}
                Substituindo $y = 2$ em $x + y = 1$
                \begin{equation*}
                    x + 2 = 1 \leadsto x = -1.
                \end{equation*}
                Substituindo em (\ref{eq:gradcond-1-q4}): $\lambda_1 = -2$, o que viola a restrição dual. Portanto, o caso não é viável.
                \item $\lambda_1 > 0$, $\lambda_2 = 0$, $\lambda_3 > 0$
                A partir de (\ref{eq:folga1-q4}) e (\ref{eq:folga2-q4}):
                \begin{equation*}
                    \begin{aligned}
                        & 1 - x - y = 0 \leadsto x + y = 1\\
                        & x - y^2 = 0 \leadsto x = y^2.
                    \end{aligned}
                \end{equation*}
                Substituindo $x = y^2$ em $x + y = 1$:
                \begin{equation*}
                    y^2 + y - 1 = 0 \leadsto y = \cfrac{-1 \pm \sqrt{5}}{2}.
                \end{equation*}
                A solução positiva é 
                \begin{equation*}
                    y = \cfrac{-1 + \sqrt{5}}{2} \approx 0.618.
                \end{equation*}
                Dessa forma, $x = y^2 \approx 0.382$, o que está de acordo com a restrição primária $y -2 = -1.382 \leq 0$.

                Substituindo em (\ref{eq:gradcond-1-q4}) e (\ref{eq:gradcond-2-q4}):
                \begin{equation*}
                    \begin{aligned}
                        & 2x - \lambda_1 + \lambda_3 = 0 \leadsto \lambda_1 - \lambda_3 = 0.764\\
                        & 2y - \lambda_1 + 2\lambda_3y = 0 \leadsto \lambda_1 + 1.236\lambda_3 = 1.236.
                    \end{aligned}
                \end{equation*}
                Resolvendo o sistema, temos que $\lambda_3 \approx -0.211$ e $\lambda_1 = 0.975$. Uma vez que ambos $\lambda_1$ e $\lambda_2$ atendem às restrições duais, o caso é viável.
                \item $\lambda_1 = 0$, $\lambda_2 > 0$, $\lambda_3 > 0$
                A partir de ($\ref{eq:folga2-q4}$) e ($\ref{eq:folga3-q4}$):
                \begin{equation*}
                    \begin{aligned}
                        & y - 2 = 0 \leadsto y = 2\\
                        & x - y^2 = 0 \leadsto x = 4.
                    \end{aligned}
                \end{equation*}
                Substituindo em (\ref{eq:gradcond-1-q4}), $\lambda_3 = -8$, o que viola a restrição dual. Portanto, o caso não é viável. 
                \item $\lambda_1 > 0$, $\lambda_2 > 0$, $\lambda_3 > 0$
                A partir de (\ref{eq:folga1-q4}), (\ref{eq:folga2-q4}) e (\ref{eq:folga3-q4}):
                \begin{equation*}
                    \begin{aligned}
                        & 1 - x - y = 0 \leadsto x + y = 1\\
                        & y - 2 = 0 \leadsto y = 2\\
                        & x - y^2 = 0 \leadsto x = 4.
                    \end{aligned}
                \end{equation*}
                Substituindo $y = 2$ em $x + y = 1$:
                \begin{equation*}
                    x + 2 = 1 \leadsto x = -1,
                \end{equation*}
                \noindent porém isso contradiz $x = 4$. Portanto, o caso não é viável.
            \end{enumerate}
            Para o único caso viável, 
            \begin{equation*}
                x \approx 0.382, \quad y = \approx -0.618.
            \end{equation*}
            Os multiplicadores de Lagrange são
            \begin{equation*}
                \lambda_1 \approx 0.975, \quad \lambda_2 = 0, \quad \lambda_3 \approx 0.211.
            \end{equation*}
            Dessa força, a função objetivo no ponto ótimo é $f(x,y) = 0.528$.
        \end{itemize}
\end{enumerate}

\end{document}
